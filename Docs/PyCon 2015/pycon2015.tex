\documentclass[hyperref={pdfpagelabels=false},compress,final]{beamer}
% \documentclass[hyperref={pdfpagelabels=false},compress,draft]{beamer}
%\let\Tiny=\tiny % get rid of some useless warnings...
\usepackage{lmodern} % also does the trick
\usepackage{default}
\usepackage[english]{babel}
% some standard options
% \usepackage{verbatim}
\usepackage{graphicx} %If you want to include postscript graphics
\usepackage{graphics}
\usepackage{color}
\usepackage{epsfig}
\usepackage{minted}
\usepackage{url}
\usepackage{float}
%\usepackage{subfigure}
\usepackage{amsmath}
\usepackage{amsfonts}
\usepackage{amssymb}
\usepackage{longtable}
\usepackage{lipsum}
% \usepackage[latin1]{inputenc}
% \usepackage{times}
\usepackage{tikz}
\usetikzlibrary{arrows,shapes,fit}
%\usepackage{lineno}
%\usepackage[bf,nooneline,format=plain,indention=.3cm]{caption}
%\setlength{\captionmargin}{10pt}
%\usepackage[letterpaper,hmargin=1in,vmargin=1.25in]{geometry}
%\frenchspacing

% some tweaking for bibliography
\usepackage{natbib}
\renewcommand{\bibsection}{\subsubsection*{\bibname }}
\def\newblock{}

% local definitions
\newcommand{\foreign}[1]{{\it #1}}
\renewcommand{\emph}[1]{{\it #1}}

\newcommand{\myhw}{0.5\textwidth}
\newcommand{\myfigw}{0.9\textwidth}
\newcommand{\myw}{0.49\textwidth}
\newcommand{\lc}{life--cycle~}

% Equation references
\renewcommand{\eqref}[1]{Eq.~(\ref{#1})}
%\renewcommand{\vec}[1]{{\bf #1}}
% Abbreviations
\newcommand{\nn}{$N^2$}
\newcommand{\nnm}{$N^2_{max}$}
\newcommand{\zet}{$\zeta$}
\newcommand{\zetp}{$\zeta_{TP}$}
\newcommand{\dze}{$\Delta\zeta_{TP}$}
\newcommand{\htp}{$h_{TP}$}
\newcommand{\dttp}{$\Delta T_{TP}$}
\newcommand{\p}{$p$}
\renewcommand{\t}{$T$}
\renewcommand{\theta}{$\theta$}
% Directions
\newcommand{\degN}{\ensuremath{^\circ}N}
\newcommand{\degS}{\ensuremath{^\circ}S}
\newcommand{\degE}{\ensuremath{^\circ}E}
\newcommand{\degW}{\ensuremath{^\circ}W}
\newcommand{\degC}{\ensuremath{^\circ}C}
% other
\renewcommand{\deg}{\ensuremath{^\circ}} % use for degree in math mode
\newcommand{\st}{\ensuremath{^{st}}}
\newcommand{\nd}{\ensuremath{^{nd}}}
\newcommand{\rd}{\ensuremath{^{rd}}}
\renewcommand{\th}{\ensuremath{^{th}}}
\usepackage{verbatim} % the comment environment is in the verbatim package
% \renewenvironment{comment}{\begin{itemize}}{\end{itemize}}
\newcommand{\inlinecomment}[1]{{\noindent\sl#1}}
% \newcommand{\app}[1]{Appendix~\ref{#1}}
\newcommand{\app}[1]{Appendix~#1}
% Units (math environment)
\newcommand{\GB}{\ensuremath{\,\mbox{GB}}}
\newcommand{\ms}{\ensuremath{\,\mbox{m\;s}^{-1}}}
\newcommand{\mmday}{\ensuremath{\,\mbox{mm\;day}^{-1}}}
\newcommand{\kgmmday}{\ensuremath{\,\mbox{kg\;m}^{-2}\,\mbox{day}^{-1}}}
\newcommand{\kgmms}{\ensuremath{\,\mbox{kg\;m}^{-2}\,\mbox{s}^{-1}}}
\newcommand{\kgms}{\ensuremath{\,\mbox{kg\;m}^{-1}\,\mbox{s}^{-1}}}
\newcommand{\mss}{\ensuremath{\,\mbox{m\;s}^{-2}}}
\newcommand{\pvu}{\ensuremath{\,\mbox{PVU}}}
\newcommand{\kkm}{\ensuremath{\,\mbox{K\;km}^{-1}}}
\newcommand{\m}{\ensuremath{\,\mbox{m}}}
\newcommand{\s}{\ensuremath{\,\mbox{s}}}
\newcommand{\Kkm}{\ensuremath{\,\mbox{K\;km}^{-1}}}
\newcommand{\km}{\ensuremath{\,\mbox{km}}}
\newcommand{\K}{\ensuremath{\,\mbox{K}}}
\newcommand{\hPa}{\ensuremath{\,\mbox{hPa}}}
\newcommand{\mssq}{\ensuremath{\,\mbox{m\;s}^2}}
\newcommand{\JKmol}{\ensuremath{\,\mbox{J\;K}^{-1}\,\mbox{mol}^{-1}}}
\newcommand{\kgmol}{\ensuremath{\,\mbox{kg\;mol}^{-1}}}
\newcommand{\ssq}{\ensuremath{\times{}10^{-4}\,\mbox{s}^{-2}}}
% math-environment abbreviations
\newcommand{\ddz}[1]{\ensuremath{\frac{\partial #1}{\partial z}}}
\newcommand{\ddp}[1]{\ensuremath{\frac{\partial #1}{\partial p}}}

%
% New Environments
%
\newenvironment{myBox}[3][shadow=true]%
{\begin{center} \begin{minipage}{#2} \begin{beamerboxesrounded}[#1]{#3} \smallskip}%
{\smallskip \end{beamerboxesrounded} \end{minipage} \end{center}}
%
% Hyphenation
\hyphenation{tro-po-pau-se}
\hyphenation{tro-po-pau-ses}
\hyphenation{pa-ra-me-ter-i-za-ti-ons}
\hyphenation{ra-di-o-son-de}
\hyphenation{me-ri-di-o-nal}
\hyphenation{a-na-ly-sis}
\hyphenation{ba-ro-cli-nic}
\hyphenation{geo-stro-phic}

%\DeclareMathOperator{\ln}{ln}
% increase line spacing by a factor of 1.5
%\renewcommand{\baselinestretch}{1.5}

\mode<presentation>{
\usetheme{Malmoe} % Malmoe Frankfurt
\usecolortheme{whale}
\usecolortheme{orchid}
\setbeamertemplate{bibliography item}[text]
\setbeamercovered{transparent}
}

\DeclareGraphicsExtensions{.png,.pdf,.jpg} %.pdf,.png,.jpg
\graphicspath{{figures/fullsize/}}
% \graphicspath{{figures/small/}}

\title[Orchestrating a Climate Modeling Pipeline]{Orchestrating a Climate Modeling Data Pipeline using Python and RAM Disk}
\subtitle{A Brief Overview of the WRF~Tools Package}
\author[\href{http://www.physics.utoronto.ca/~aerler/}{Andre R. Erler} (\href{mailto:A.R.Erler@gmail.com}{A.R.Erler@gmail.com})]{Andre R. Erler}
\institute{PyCon Canada}
\date{November $7^{th}$, 2015}
%\logo{\includegraphics[scale=0.15]{HD3}}

% \AtBeginSection[]
% {
%    \begin{frame}{Outline}
%        \tableofcontents[currentsection]
%    \end{frame}
% }

% *************************************************************************************************
\providecommand\thispdfpagelabel[1]{}

\begin{document}

% \section*{\ }

\begin{frame}
\titlepage
\end{frame}

\begin{frame}{Outline}
\tableofcontents%[pausesections]
\end{frame}

\section[Introduction]{Introduction: Climate Modeling}
% \begin{frame}{Outline}
% \tableofcontents[currentsection]
% \end{frame}


% ################################################################################################


\subsection*{Global Models}

\begin{frame}{\hspace*{0.5\textwidth}Global Climate Models}
 \begin{columns}
   \begin{column}{0.4\textwidth}
     \vspace*{-1cm}\\%\hspace*{-0.5cm}
     \includegraphics[width=\textwidth]{AR41projection}\\
     {\footnotesize IPCC AR4 (2007) projections for global surface temperature under different scenarios.} \\ \smallskip
   \begin{myBox}[shadow=true]{0.85\textwidth}{}
     \small Global Climate Models are the main tool to predict climate change.
   \end{myBox}
   \end{column}
   \begin{column}{0.7\textwidth}
%      \vspace*{0.0cm}\\%\hspace*{-2cm}
     {\footnotesize Climate models compute energy, mass, and momentum fluxes on a relatively coarse computational grid. \smallskip \\
     Schematic of a Global Climate Model (GCM):} \smallskip \\
     \includegraphics[width=\textwidth]{climateSystem2}\\
     {\tiny Center for Multiscale Modeling of Atmospheric Processes, CSU}
   \end{column}
 \end{columns}
\end{frame}

\subsection{Regional Models}

\begin{frame}{Regional Climate Models}
  \begin{columns}
    \begin{column}{0.5\textwidth}
      \vspace*{0.2cm}\\%\hspace*{-0.5cm}
      {\small GCM resolution is coarse and many regional details are not resolved (e.g. the Rocky Mountains and the Great Lakes).}
      \includegraphics[width=2\textwidth]{giorgi_rcm}\\
      \vspace{-0.25cm}{\tiny Giorgi (2006)}
    \end{column}
    \begin{column}{0.5\textwidth}
      \vspace*{-1.cm}\\%\hspace*{-0.5cm}
      \onslide<2->{
      \begin{myBox}[shadow=true]{0.85\textwidth}{Regional Impacts}
        \small Regional impacts of Climate Change are modeled with high-resolution regional climate models (RCM).
      \end{myBox}}
      \vspace*{4cm}
      {\small Regional models simulate a small area at much higher resolution ($\times 10$).}
    \end{column}
  \end{columns}
\end{frame}

\begin{frame}{}
  \begin{columns}
    \begin{column}{0.5\textwidth}
      \small
      \begin{myBox}[shadow=true]{0.92\textwidth}{Global Model: CESM}
        \small The Community Earth System Model is used as driving model.
      \end{myBox}
%       \\ \vspace*{.25cm}
      \includegraphics[width=\textwidth]{Ts_annual_ortho_ndl} \\
      {\footnotesize Average Surface Temperature and Outline of the WRF Domain}
    \end{column}
    \begin{column}{0.5\textwidth}
      %\vspace*{-1cm}
      \small
      \includegraphics[width=\textwidth]{zs_hidef_noWRF} \\ \smallskip
      {\footnotesize Topography of Western Canada.\\
      Left: CESM at $\sim$80\km{}\\
      Right: WRF at 10\km{} }
      \begin{myBox}[shadow=true]{0.96\textwidth}{Regional Model: WRF}
        \small The Weather Research and Forecast model is our regional model.
      \end{myBox}
    \end{column}
  \end{columns}
\end{frame}

\begin{frame}
  \begin{columns}
    \begin{column}{0.55\textwidth}
      \includegraphics[width=\textwidth]{worf.jpg}
    \end{column}
    \begin{column}{0.35\textwidth}
      {WRF is actually pronounced ``Worf'',\\ like Lt. Worf in \\\textit{Star Trek: The Next Generation} (left)}
      \\ \bigskip \medskip
      {\small The WRF model is a limited-area numerical weather prediction model developed by the National Center for Atmospheric Research}
%       \underline{\bf \large The WRF Model} \\ \bigskip \medskip
    \end{column}
  \end{columns}
\end{frame}


% ################################################################################################


\section[\ The Pipeline]{The Pre-processing Pipeline}

\begin{frame}{\hspace*{0.3cm} Running the Regional \\\hspace*{0.3cm} Climate Model (WRF)}
  \begin{columns}
    \begin{column}{0.5\textwidth}
      \begin{itemize}
        \item<2-> The coupling process between GCM and RCM is ``off-line'' (asynchronous) \smallskip
        \item<2-> A pre-processing system converts GCM output into RCM (wrf-)input files \bigskip
        \item<3-> A RCM simulation is split into $\sim200$ separate jobs \smallskip
        \item<3-> The RCM runs continuously, each job submitting the next \bigskip
      \end{itemize}
    \end{column}
    \begin{column}{0.5\textwidth}
      \vspace*{-1.5cm}
      \begin{myBox}{0.9\textwidth}{The WRF Tools Package}
        \textbf{\color{teal}Python}
        \begin{itemize}
          \item<2> Run pre-processing\\ tool chain (WPS)
          \item<3> Initialize WRF jobs
          \item<4> Run post-processing \smallskip
        \end{itemize}
        {\color{purple}Shell Script}
        \begin{itemize}
          \item<2> Submit pre-processing
          \item<3> Run the WRF job,\\ submit next job
          \item<4> Archiving to tape \medskip
        \end{itemize}
        WRF Tools enables continuous and autonomous operation
      \end{myBox}
      %   WRF restarts from its own state, so the initialization from CESM is relatively irrelevant
    \end{column}
  \end{columns}
\end{frame}

\begin{frame}{The Data Pipeline}
  \tikzstyle{format} = [draw, fill=blue!20, rounded corners]
  \tikzstyle{medium} = [ellipse, draw, thin, fill=green!20, minimum height=2.5em]
  \begin{figure}
    \begin{tikzpicture}[node distance=2cm, auto,>=latex', thick]
      % We need to set at bounding box first. Otherwise the diagram
      % will change position for each frame.
      %     \path[use as bounding box] (-1,0) rectangle (10,-2);
      %     \draw[help lines] (0,0) grid (5,5);
      \scriptsize
      % CESM
      \uncover<1->{
        \node[format] at (0,0) (cesm1) {Global Model (CESM)};
        \draw[->, blue, ultra thick] (cesm1) -- (8,0);
        \draw[->, red, ultra thick] (cesm1) -- (0,-0.5) |- (8,-0.5);
        \path[red] (1,-0.5) edge node {asynchronous/archived output} (7,-0.5);
        \node[format, align=left] at (7,-2) (ar) {Long-term\\Archive,\\Post-\\processing,\\Analysis};
        \draw[<-, blue, ultra thick] (ar) -- (7,0);
      }
      % WPS/WRF 1
      \uncover<1->{
        \path[->] node[format, below of=cesm1] (wps1) {WPS};
        \draw[->,red,ultra thick](cesm1) edge node {offline} (wps1);
        \path[->] node[format, right of=wps1] (wrf1) {WRF};
        \draw[->,violet,ultra thick] (wps1) edge node {wrfinput} (wrf1);
        \draw[->,blue,ultra thick] (wrf1) edge node {wrfout} (ar);
      }
      % WPS/WRF 2
      \uncover<2->{
        \path[->] node[format, below of=wrf1] (wps2) {WPS};
        \draw[<-,red,ultra thick] (wps2) -- +(-0.5,0) -| (1,-0.5);
        \path[->] node[format, right of=wps2] (wrf2) {WRF};
        \draw[->,violet,ultra thick] (wps2) edge node {wrfinput} (wrf2);
        \draw[->,blue,ultra thick] (wrf2) edge node {wrfout} (ar);
        \draw[->] (wrf1) edge node {launch} (wps2);
        \draw[->] (wrf1) -- (wrf2);
      }
      % WPS/WRF 3
      \uncover<3->{
        \path[->] node[format, below of=wrf2] (wps3) {WPS};
        \draw[<-,red,ultra thick] (wps3) -- +(-0.5,0) -| (3,-0.5);
        \path[->] node[format, right of=wps3] (wrf3) {WRF};
        \draw[->,violet,ultra thick] (wps3) edge node {wrfinput} (wrf3);
        \draw[->,blue,ultra thick] (wrf3) edge node {wrfout} (ar);
        \draw[->] (wrf2) edge node {launch} (wps3);
        \draw[->] (wrf2) -- (wrf3);
      }
      % geogrid
      \uncover<4->{
        \node[format] at (0.,-6) (geo) {geogrid};
        \draw[->,teal,thick] (geo) -- (wps1);
        \draw[->,teal,thick] (geo) edge node [align=left,above left] {static\\ data} (wps2);
        \draw[->,teal,thick] (geo) -- (wps3);
      }
    \end{tikzpicture}
  \end{figure}
\end{frame}


% ################################################################################################

\subsection{The WRF Tools Package}

% \begin{frame}{\hspace*{0.8cm} WPS: \textsc{\huge \color{purple} Fortran} \\\hspace*{0.8cm} Legacy Tools}
\begin{frame}{\hspace*{0.5cm} WPS: A Collection of \textsc{\huge \color{purple} Fortran} Legacy Tools}
  \begin{columns}
    \begin{column}{0.4\textwidth}
      \begin{myBox}{\textwidth}{WPS Components}
        \small
        \begin{enumerate}
          \item {\color{teal} \tt geogrid.exe} \\ static\,/\,geographic data
          \item {\color{red} \tt ungrib.exe / unccsm.exe} \\
          convert driving data to WRF IM Format
          \item {\color{blue} \tt metgrid.exe} \\ interpolate to WRF grid
          \item {\color{violet} \tt real.exe} \\ generate boundary condition files
        \end{enumerate}
      \end{myBox}
    \end{column}
    \begin{column}{0.5\textwidth}
      \textsc{\Large\color{purple} Fortran} legacy tools read from and write to temporary files:
      \begin{itemize}
        \item Strongly I/O limited in a HPC cluster environment
      \end{itemize}
      \medskip
      \onslide<2->{\begin{myBox}{0.8\textwidth}{The Solution (on Linux)}
        Run on RAM-disk!
        \begin{itemize}
          \item speedup $\sim\times10$
          \item requires 64\,GB RAM
        \end{itemize}
        Using Python driver script
      \end{myBox}}
    \end{column}
  \end{columns}
\end{frame}

\section[PyWPS]{Using Python to Drive the Pipeline}

\subsection*{Overview} % and parallelization
\definecolor{lightgray}{rgb}{0.85,0.9,0.9}
\definecolor{darkgreen}{rgb}{0.0,0.4,0.2}
\setbeamercolor{upcol}{fg=white,bg=darkgreen}
\setbeamercolor{lowcol}{fg=black,bg=lightgray}

\begin{frame}{\hspace*{2.5cm}{\color{teal}PyWPS}: A Driver Module for WPS}
  \begin{columns}
    \begin{column}{0.6\textwidth}
      \begin{itemize}
        \item Collect required input data from GCM archive \smallskip
        \item Run applicable pre-processing tools on RAM disk \smallskip
        \item Assemble WRF input files
      \end{itemize}
      \medskip
      \onslide<2->{\begin{myBox}[upper=upcol,lower=lowcol,shadow=true]{0.8\textwidth}{Why Python?}
          \begin{itemize}
            \item Easier with complex logic \smallskip
            \item Classes for different datasets/GCMs
          \end{itemize}
        \end{myBox}}
    \end{column}
    \begin{column}{0.4\textwidth}
      \begin{myBox}{\textwidth}{PyWPS Imports}
        \small
        \begin{itemize}
          \item \texttt{\color{purple} multiprocessing} for parallelization
          \item \texttt{re} to find input files
          \item \texttt{fileinput}, \texttt{sys} to edit configurations files
          \item \texttt{\color{purple} subprocess} to launch \textsc{Fortran} tools
          \item \texttt{shutils}, \texttt{os} to handle temporary files
        \end{itemize}
      \end{myBox}
    \end{column}
  \end{columns}
\end{frame}

\subsection{The Tool Chain}

\begin{frame}{\hspace*{.7cm}{\color{teal}PyWPS}: The Program Flow \& Parallelization}
  \begin{figure}
    \hspace*{-1.1cm}
    \tikzstyle{core} = [draw, thin, fill=blue!40, rounded corners, align=left]
    \tikzstyle{allfiles} = [draw, thin, fill=purple!20, rounded corners, align=left]
    \tikzstyle{regex} = [draw, thin, fill=teal, fill opacity=0.15, rounded corners, align=center]
    \tikzstyle{jobfiles} = [draw, thin, fill=red!20, rounded corners, align=left]
    \tikzstyle{wps} = [draw, thin, fill=purple, fill opacity=0.15, rounded corners, align=center]
    \tikzstyle{wrfinput} = [draw, thin, fill=violet!20, rounded corners, align=center]
    \begin{tikzpicture}[auto,>=latex']
      % We need to set at bounding box first. Otherwise the diagram
      % will change position for each frame.
      %     \path[use as bounding box] (-1,0) rectangle (10,-2);
      %     \draw[help lines] (0,0) grid (5,5);
      \scriptsize
      % select files
      \uncover<1->{
        \node[allfiles] at (0.25,1.5) (allfiles) {All CESM\\Output files:\\...};
        \node[jobfiles] at (5,1.5) (jobfiles) {Selected CESM\\Output files:\\only current job};
        \node[core] at (2.5,3) (core10) {core 0}; \draw[->,black,thick] (allfiles) -- (core10); \draw[->,black,thick] (core10) -- (jobfiles);
        \node[core] at (2.5,2) (core11) {core 1}; \draw[->,black,thick] (allfiles) -- (core11); \draw[->,black,thick] (core11) -- (jobfiles);
        \node[core] at (2.5,1) (core12) {core 2}; \draw[->,black,thick] (allfiles) -- (core12); \draw[->,black,thick] (core12) -- (jobfiles);
        \node[core] at (2.5,0) (core13) {core 3}; \draw[->,black,thick] (allfiles) -- (core13); \draw[->,black,thick] (core13) -- (jobfiles);
        \node[]     at (2.5,3.5) (regex) {Select Files};
        \node[regex,fit=(regex) (core10) (core11) (core12) (core13)] {}; }
      % process files
      \uncover<2->{
        \node[wrfinput] at (10.5,1.5) (wrfinput) {WRF input\\ files};
        \node[core] at (8.0,3) (core20) {core 0}; \draw[->,black,thick] (jobfiles) -- (core20); \draw[->,black,thick] (core20) -- (wrfinput);
        \node[core] at (8.0,2) (core21) {core 1}; \draw[->,black,thick] (jobfiles) -- (core21); \draw[->,black,thick] (core21) -- (wrfinput);
        \node[core] at (8.0,1) (core22) {core 2}; \draw[->,black,thick] (jobfiles) -- (core22); \draw[->,black,thick] (core22) -- (wrfinput);
        \node[core] at (8.0,0) (core23) {core 3}; \draw[->,black,thick] (jobfiles) -- (core23); \draw[->,black,thick] (core23) -- (wrfinput);
        \node[]     at (8.0,3.5) (wps) {WPS Tool Chain};
        \node[wps,fit=(wps) (core20) (core21) (core22) (core23)] {};}
    \end{tikzpicture}
  \end{figure}
  \bigskip
%   \hspace*{-0.5cm}
  \onslide<3->{
  {\Large \usebeamercolor[fg]{frametitle} The {\color{purple}WPS Tool Chain}:}
  \begin{figure}
    \tikzstyle{format} = [draw, thin, fill=blue!20, rounded corners, align=center]
    \hspace*{-0.25cm}\vspace*{-0.25cm}
    \begin{tikzpicture}[auto,>=latex']
      \scriptsize
      \node[format] at (0,0) (cesm) {CESM\\output};
      \node[format] at (3.5,0) (wrfim) {WRF inter-\\mediate file};
      \draw[->,red,ultra thick] (cesm) edge node [align=above] {ungrib.exe} (wrfim);
      \node[format] at (7,0) (metgrid) {metgrid\\file};
      \draw[->,blue,ultra thick] (wrfim) edge node [align=above] {metgrid.exe} (metgrid);
      \node[format] at (10,0) (real) {wrfinput};
      \draw[->,violet,ultra thick] (metgrid) edge node [align=above] {real.exe} (real);
    \end{tikzpicture}
  \end{figure}}
\end{frame}

\subsection{The Class Structure}

\begin{frame}[fragile=singleslide]{The Class Structure}
  \begin{columns}
    \begin{column}{0.5\textwidth}
      Dataset/GCM specific parameters: \smallskip
      \begin{itemize}
        \item Input file types/names\smallskip
        \item Interpolation tables/grid \smallskip
        \item Variables / frequency \smallskip
      \end{itemize}
      \onslide<2->{
      \begin{myBox}{0.9\textwidth}{Multiple Datasets}
        \begin{itemize}
          \item \textit{\color{purple} Inheritance} for common procedures \smallskip
          \item \textit{\color{teal} Polymorphism} for different procedures
        \end{itemize}
      \end{myBox}}
    \end{column}
    \begin{column}{0.5\textwidth}
      \vspace*{-1.25cm}
      \begin{myBox}{1.0\textwidth}{}
%         \renewcommand{\theFancyVerbLine}{\sffamily\textcolor[rgb]{0.5,0.5,0.5}{\scriptsize\arabic{FancyVerbLine}}}
        \footnotesize
        \begin{minted}{python}
class Dataset(object):
  prefix = ''  # file prefix
  vtable = 'Vtable'
  gribname = 'GRIBFILE' # input
  ungrib_exe = 'ungrib.exe'
  ungrib_log = 'ungrib.exe.log'
  ...
  def __init__(self, ...):
    # type checking
    ...
  def setup(self, src, ...):
    ...
  def cleanup(self, tgt):
    ...
  def extractDate(self, fname):
    # match valid filenames
    ...
  def ungrib(self, date, mytag):
    # generate file for metgrid
    ...
        \end{minted}
      \end{myBox}
    \end{column}
  \end{columns}
\end{frame}

\begin{frame}[fragile=singleslide]{The Class Structure}
  \begin{columns}
    \begin{column}{0.5\textwidth}
      Dataset/GCM specific parameters: \smallskip
      \begin{itemize}
        \item Input file types/names\smallskip
        \item Interpolation tables/grid \smallskip
        \item Variables / frequency \smallskip
      \end{itemize}
        \begin{myBox}{0.9\textwidth}{Multiple Datasets}
          \begin{itemize}
            \item \textit{\color{purple} Inheritance} for common procedures \smallskip
            \item \textit{\color{teal} Polymorphism} for different procedures
          \end{itemize}
        \end{myBox}
    \end{column}
    \begin{column}{0.5\textwidth}
      \vspace*{-1.25cm}
      \begin{myBox}{1.0\textwidth}{}
        %         \renewcommand{\theFancyVerbLine}{\sffamily\textcolor[rgb]{0.5,0.5,0.5}{\scriptsize\arabic{FancyVerbLine}}}
        \footnotesize
        \begin{minted}{python}
          class Dataset(object):
          prefix = ''  # file prefix
          vtable = 'Vtable'
          gribname = 'GRIBFILE' # input
          ungrib_exe = 'ungrib.exe'
          ungrib_log = 'ungrib.exe.log'
          ...
          def __init__(self, ...):
          # type checking
          ...
          def setup(self, src, ...):
          ...
          def cleanup(self, tgt):
          ...
          def extractDate(self, fname):
          # match valid filenames
          ...
          def ungrib(self, date, mytag):
          # generate file for metgrid
          ...
        \end{minted}
      \end{myBox}
    \end{column}
  \end{columns}
\end{frame}

\section[Conclusion]{Concluding Remarks}

\begin{frame}{Summary \& Conclusion}

  {\Large \color{teal} \textbf{Python} }\smallskip
  \begin{itemize}
    \item Use Python for flow control (manage legacy tools) \medskip
    \item Parallelization relatively easy (within one node) \medskip
    \item Class structure is versatile and makes maintenance easier \medskip
  \end{itemize}
  \bigskip
  {\Large \color{purple} {RAM}-disk }\smallskip
  \begin{itemize}
    \item Scientific Programming: dealing with legacy tools\\
          Often in \textsc{Fortran}, often relying on disk I/O \medskip
    \item Use RAM-disk to avoid unnecessary disk I/O \medskip
  \end{itemize}
\end{frame}

\begin{frame}
 \vfill
 \begin{center}
 \LARGE
 Thank You!\hspace{1.5cm}  $\sim$ \hspace{1.5cm} Questions?
 \end{center}
 \vfill
\end{frame}

% ################################################################################################

\begin{frame}{\Large List of Publications using WRF~Tools}
  \medskip
  \begin{itemize}
    \scriptsize
    \item \textbf{Erler, Andre R.}, W. Richard Peltier, Marc d'Orgeville (under review),
    Projected Changes in Hydro-Climatic Extremes for Western Canada,
    Journal of Climate. \bigskip

    \item Marc d'Orgeville, W. Richard Peltier, \textbf{Andre R. Erler} (accepted),
    Uncertainty in Future Summer Precipitation on the Great Lakes Basin due to Drought in the South-Western US,
    Journal of Geophysical Research. \bigskip

    \item \href{http://journals.ametsoc.org/doi/abs/10.1175/JCLI-D-14-00174.1}
    {\textbf{Erler, Andre R.}, W. Richard Peltier, Marc d'Orgeville, 2015,
      Dynamically Downscaled High Resolution Hydro-Climate Projections for Western Canada,
      Journal of Climate.} \bigskip

    \item \href{http://onlinelibrary.wiley.com/enhanced/doi/10.1002/2014JD021855/}
    {Marc d'Orgeville, W. Richard Peltier, \textbf{Andre R. Erler}, Jonathan Gula, 2014,
      Climate change impacts on Great Lakes Basin precipitation extremes,
      Journal of Geophysical Research.} \bigskip
  \end{itemize}
  %   \smallskip

\end{frame}


% *************************************************************************************************
%  -----------------------------------------------------------------------------------------------
% *************************************************************************************************

% \begin{comment}

\appendix

\subsection*{Experimental Setup}

\begin{frame}{Regional Climate Projections}
  \begin{columns}[T]
    \begin{column}{0.6\textwidth}
      \underline{\Large Experimental Setup}\medskip
      \small
      \begin{itemize}
        \item GCM \& RCM run for 15\,years \\ (model time) \smallskip
        \begin{itemize}
          \item Historical (1979\,-\,1994) \smallskip
          \item Mid-21\st-Century (2045-2060) \smallskip
          \item End-21\st-Century (2085-2100)
        \end{itemize} \smallskip %\pause
        \item GCM \& RCM use RCP\,8.5 GHG concentration scenarios
        \item RCM runs with different physical parameterizations \smallskip %\pause
        \item Both models run in an initial condition ensemble with 4 members each
      \end{itemize}
    \end{column}
    \begin{column}{0.4\textwidth}
      \vspace*{-.8cm} \hspace*{-.8cm}
      \includegraphics[width=1.3\textwidth]{AR41projection.png}
      \begin{myBox}{\textwidth}{}
        {\footnotesize \mbox{IPCC AR4 climate projections} based on different scenarios; %  GHG emission
          the RCP\,8.5 is very similar to the older A2 scenario}
      \end{myBox}
      %   \begin{myBox}{\textwidth}{}
      %   \end{myBox}
      %   \includegraphics[width=1.1\textwidth]{ARB.png}\\ \medskip \small
      %   The Athabasca River Basin hydrological modeling project is one application for our data.
      %   Groundwater (top) and surface water (bottom).
    \end{column}
  \end{columns}
\end{frame}

% large domain map
\begin{frame}
  \begin{center}
    \vspace*{-0.75cm}
%     \includegraphics<1>[height=0.9\textheight]{zs_topo_arb2}
    \includegraphics[height=0.9\textheight]{zs_topo_arb2_basins}
  \end{center}
\end{frame}


\subsection*{Summary of Results}

\begin{frame}{\LARGE Summary of Results}
  \begin{columns}[T]
    \begin{column}{0.6\textwidth}
      \smallskip
      \begin{itemize}
        %         \item<1-> WRF improves significantly over CESM, but are broadly consistent \smallskip
        \item<1-> Significant increase in winter precipitation (extremes, $\sim$30\%) \medskip
        \item<1-> Small increase in summer, but more increase in evaporation
      \end{itemize}
      %  \smallskip
      {\small
        \onslide<1->{
          \begin{myBox}[shadow=true]{.9\textwidth}{Hydrological Impacts}
            \begin{itemize}
              \item Climate change impacts in ARB/Alberta likely benign\medskip
              \item 50\% reduction in peak snow\-melt and spring runoff in FRB/BC...
              \item ... but increased flood risk due to precipitation extremes in fall
            \end{itemize}
          \end{myBox}}}
    \end{column}
    \begin{column}{0.4\textwidth}
      \vspace*{-1.2cm} \hspace*{-.6cm}
      \includegraphics[width=5.5cm]{aSM_frac_jas_sum}\\ \smallskip
      \parbox{1.\textwidth}{\small End-century soil moisture changes in late summer}
      \smallskip
      \begin{itemize}
        \item<1-> Late summer drying west of Continental Divide, but not east
      \end{itemize}
    \end{column}
  \end{columns} %\small
\end{frame}

% \end{comment}

\end{document}
